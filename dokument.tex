%╔════════════════════════════╗
%║	  Szablon dostosował	  ║
%║	mgr inż. Dawid Kotlarski  ║
%║		  06.10.2024		  ║
%╚════════════════════════════╝


\documentclass[12pt,twoside,a4paper,openany]{article}

    \input{preambula_pakiety.tex}
    \input{preambula_ustawienia.tex}

    %polecenia zdefiniowane w pakiecie strona_tytulowa.sty
    \title{Zmienić tytuł [IMPORTANT]}		%...Wpisać nazwę projektu...
    \author{Imie Nazwisko}
    \author{Arkadiusz Ryczek, Maciej Wójs,  }
    \authorI{Dominik Żuchowicz, Hubert Zięba}
    \authorII{Szymon Wolski, Filip Wąchała, Jakub \textit{Józef} Tokarczyk}		%jeśli są dwie osoby w projekcie to zostawiamy:    \authorII{}
		
	\uczelnia{AKADEMIA NAUK STOSOWANYCH \\W NOWYM SĄCZU}
    \instytut{Wydział Nauk Inżynieryjnych}
    \kierunek{Katedra Informatyki}
    \praca{DOKUMENTACJA PROJEKTOWA}
    \przedmiot{BEZPIECZEŃSTWO SYSTEMÓW INFORMATYCZNYCH}
    \prowadzacy{mgr inż. Jacek Kaleta}
    \rok{2025}

%definicja składni mikrotik
\usepackage{fancyvrb}
\DefineVerbatimEnvironment{MT}{Verbatim}%
{commandchars=\+\[\],fontsize=\small,formatcom=\color{red},frame=lines,baselinestretch=1,} 
\let\mt\verb
%zakonczenie definicji składni mikrotik

\usepackage{fancyhdr}    %biblioteka do nagłówka i stopki
\begin{document}

\renewcommand{\figurename}{Rys.}    %musi byc pod \begin{document}, bo w~tym miejscu pakiet 'babel' narzuca swoje ustawienia
\renewcommand{\tablename}{Tab.}     %j.w.
\thispagestyle{empty}               %na tej stronie: brak numeru
\stronatytulowa                     %strona tytułowa tworzona przez pakiet strona_tytulowa.tex

% \input{tabela.tex} % tabela z danymi projektu

\pagestyle{fancy}
\newpage

%formatowanie spisu treści i~nagłówków
\renewcommand{\cftbeforesecskip}{8pt}
\renewcommand{\cftsecafterpnum}{\vskip 8pt}
\renewcommand{\cftparskip}{3pt}
\renewcommand{\cfttoctitlefont}{\Large\bfseries\sffamily}
\renewcommand{\cftsecfont}{\bfseries\sffamily}
\renewcommand{\cftsubsecfont}{\sffamily}
\renewcommand{\cftsubsubsecfont}{\sffamily}
\renewcommand{\cftparafont}{\sffamily}
%koniec formatowania spisu treści i nagłówków

\tableofcontents    %spis treści
\thispagestyle{fancy}
\newpage

%%%%%%%%%%%%%%%%%%% treść główna dokumentu %%%%%%%%%%%%%%%%%%%%%%%%%
\section{Założenia projektowe – wymagania}

\end{document}
