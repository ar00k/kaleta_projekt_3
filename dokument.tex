%╔════════════════════════════╗
%║	  Szablon dostosował	  ║
%║	mgr inż. Dawid Kotlarski  ║
%║		  06.10.2024		  ║
%╚════════════════════════════╝


\documentclass[12pt,twoside,a4paper,openany]{article}

    \input{preambula_pakiety.tex}
    \input{preambula_ustawienia.tex}

    %polecenia zdefiniowane w pakiecie strona_tytulowa.sty
    \title{Konfiguracja Site-to-Site VPN w Cisco IOS}		%...Wpisać nazwę projektu...
    \author{Imie Nazwisko}
    \author{Arkadiusz Ryczek, Maciej Wójs,  }
    \authorI{Dominik Żuchowicz, Hubert Zięba}
    \authorII{Szymon Wolski, Filip Wąchała, Jakub \textit{Józef} Tokarczyk}		%jeśli są dwie osoby w projekcie to zostawiamy:    \authorII{}
		
	\uczelnia{AKADEMIA NAUK STOSOWANYCH \\W NOWYM SĄCZU}
    \instytut{Wydział Nauk Inżynieryjnych}
    \kierunek{Katedra Informatyki}
    \praca{DOKUMENTACJA PROJEKTOWA}
    \przedmiot{BEZPIECZEŃSTWO SYSTEMÓW INFORMATYCZNYCH}
    \prowadzacy{mgr inż. Jacek Kaleta}
    \rok{2025}

    \definecolor{codegreen}{rgb}{0,0.6,0}
    \definecolor{codegray}{rgb}{0.5,0.5,0.5}
    \definecolor{codepurple}{rgb}{0.58,0,0.82}
    \definecolor{backcolour}{rgb}{0.95,0.95,0.92}

    \lstdefinestyle{mystyle}{
        backgroundcolor=\color{backcolour},   
        commentstyle=\color{codegreen},
        keywordstyle=\color{magenta},
        numberstyle=\tiny\color{codegray},
        stringstyle=\color{codepurple},
        basicstyle=\ttfamily\footnotesize,
        breakatwhitespace=false,         
        breaklines=true,                 
        captionpos=b,                    
        keepspaces=true,                 
        numbers=left,                    
        numbersep=5pt,                  
        showspaces=false,                
        showstringspaces=false,
        showtabs=false,                  
        tabsize=2,
        morecomment=[l]{!}
    }

    \lstset{style=mystyle}
%definicja składni mikrotik
\usepackage{fancyvrb}
\DefineVerbatimEnvironment{MT}{Verbatim}%
{commandchars=\+\[\],fontsize=\small,formatcom=\color{red},frame=lines,baselinestretch=1,} 
\let\mt\verb
%zakonczenie definicji składni mikrotik

\usepackage{fancyhdr}    %biblioteka do nagłówka i stopki
\begin{document}

\renewcommand{\figurename}{Rys.}    %musi byc pod \begin{document}, bo w~tym miejscu pakiet 'babel' narzuca swoje ustawienia
\renewcommand{\tablename}{Tab.}     %j.w.
\thispagestyle{empty}               %na tej stronie: brak numeru
\stronatytulowa                     %strona tytułowa tworzona przez pakiet strona_tytulowa.tex

% \input{tabela.tex} % tabela z danymi projektu

\pagestyle{fancy}
\newpage

%formatowanie spisu treści i~nagłówków
\renewcommand{\cftbeforesecskip}{8pt}
\renewcommand{\cftsecafterpnum}{\vskip 8pt}
\renewcommand{\cftparskip}{3pt}
\renewcommand{\cfttoctitlefont}{\Large\bfseries\sffamily}
\renewcommand{\cftsecfont}{\bfseries\sffamily}
\renewcommand{\cftsubsecfont}{\sffamily}
\renewcommand{\cftsubsubsecfont}{\sffamily}
\renewcommand{\cftparafont}{\sffamily}
%koniec formatowania spisu treści i nagłówków

\tableofcontents    %spis treści
\thispagestyle{fancy}
\newpage

%%%%%%%%%%%%%%%%%%% treść główna dokumentu %%%%%%%%%%%%%%%%%%%%%%%%%
\section{Wymagania projektu}
\begin{itemize}
\item \textbf{Topologia:} Dwa oddzielne lokacje (Site A i Site B) połączone przez publiczną sieć ISP
\item \textbf{Urządzenia:} Dwa routery Cisco (R1 i R3) pełniące rolę bram VPN
\item \textbf{Wymagania bezpieczeństwa:}
  \begin{itemize}
  \item Szyfrowanie ruchu między lokacjami
  \item Uwierzytelnianie urządzeń
  \item Integralność przesyłanych danych
  \end{itemize}
\item \textbf{Wymagania funkcjonalne:}
  \begin{itemize}
  \item Tunel IPSec między R1 (10.1.1.1) i R3 (10.2.2.1)
  \item Komunikacja między sieciami LAN (192.168.1.0/24 i 192.168.3.0/24)
  \item Wykorzystanie pre-shared key do uwierzytelniania
  \end{itemize}
\end{itemize}

\section{Topologia sieci}
\begin{figure}[h!]
\centering
\includegraphics[width=0.8\textwidth]{rys/Screenshot From 2025-06-09 18-29-04.png}
\caption{Topologia połączenia VPN między Site A i Site B}
\label{fig:vpn_topology}
\end{figure}
\textbf{Uwaga:} Router R2 pełni rolę symulowanego dostawcy internetowego (ISP) i nie uczestniczy w konfiguracji VPN.

\section{Tabela adresacji IP}
\begin{figure}[h!]
    \centering
    \includegraphics[width=0.95\textwidth]{rys/adresy.png}
    \caption{Tabela adresacji IP dla urządzeń w projekcie}
    \label{fig:adresacja_ip}
\end{figure}

\section{Konfiguracja routera R1}
\begin{lstlisting}[caption={Konfiguracja podstawowa i VPN na R1}]
! Konfiguracja interfejsów
! Przypisanie adresu IP do interfejsu GigabitEthernet0/1 (sieć LAN Site A)
interface GigabitEthernet0/1
 ip address 192.168.1.1 255.255.255.0
 no shutdown! Aktywacja interfejsu
!
! Przypisanie adresu IP do interfejsu Serial0/0/0 (połączenie z ISP)
interface Serial0/0/0
 ip address 10.1.1.1 255.255.255.252
 no shutdown! Aktywacja interfejsu

! Routing statyczny
! Definicja trasy domyślnej kierującej ruch do ISP
ip route 0.0.0.0 0.0.0.0 10.1.1.2

! Konfiguracja ISAKMP (IKE Phase 1)
! Definicja polityki ISAKMP (Internet Security Association and Key Management Protocol)
crypto isakmp policy 10
 encr 3des            !Użycie szyfrowania 3DES
 hash md5             !Użycie funkcji skrótu MD5
 authentication pre-share !Metoda uwierzytelniania: pre-shared key
 group 2              !Grupa Diffie-Hellmana (DH) numer 2 (1024-bit)
! Definicja klucza pre-shared dla zdalnego routera (R3)
crypto isakmp key cisco123 address 10.2.2.1

! Konfiguracja transform set (IKE Phase 2)
! Definicja zestawu transformacji określającego, jak dane będą chronione
crypto ipsec transform-set MYSET esp-3des esp-md5-hmac
! esp-3des: Szyfrowanie ESP (Encapsulating Security Payload) za pomocą 3DES
! esp-md5-hmac: Uwierzytelnianie ESP za pomocą HMAC-MD5

! Definicja listy dostępu dla ruchu VPN
! Określenie, który ruch sieciowy ma być szyfrowany (z LAN Site A do LAN Site B)
ip access-list extended VPN-TRAFFIC
 permit ip 192.168.1.0 0.0.0.255 192.168.3.0 0.0.0.255

! Tworzenie i aplikacja mapy kryptograficznej
! Powiązanie wszystkich elementów konfiguracji VPN w jedną całość
crypto map MYMAP 10 ipsec-isakmp
 set peer 10.2.2.1            !Adres IP zdalnego routera VPN (R3)
 set transform-set MYSET      !Użycie zdefiniowanego zestawu transformacji
 match address VPN-TRAFFIC    !Powiązanie z listą dostępu definiującą ruch VPN

! Aplikacja mapy kryptograficznej do interfejsu wychodzącego (w kierunku ISP)
interface Serial0/0/0
 crypto map MYMAP
\end{lstlisting}

\section{Konfiguracja routera R3}
\begin{lstlisting}[caption={Konfiguracja podstawowa i VPN na R3}]
! Konfiguracja interfejsów
! Przypisanie adresu IP do interfejsu GigabitEthernet0/1 (sieć LAN Site B)
interface GigabitEthernet0/1
 ip address 192.168.3.1 255.255.255.0
 no shutdown! Aktywacja interfejsu
!
! Przypisanie adresu IP do interfejsu Serial0/0/1 (połączenie z ISP)
interface Serial0/0/1
 ip address 10.2.2.1 255.255.255.252
 no shutdown! Aktywacja interfejsu

! Routing statyczny
! Definicja trasy domyślnej kierującej ruch do ISP
ip route 0.0.0.0 0.0.0.0 10.2.2.2

! Konfiguracja ISAKMP (IKE Phase 1)
! Definicja polityki ISAKMP, musi być zgodna z konfiguracją na R1
crypto isakmp policy 10
 encr 3des
 hash md5
 authentication pre-share
 group 2
! Definicja klucza pre-shared dla zdalnego routera (R1)
crypto isakmp key cisco123 address 10.1.1.1

! Konfiguracja transform set (IKE Phase 2)
! Definicja zestawu transformacji, musi być zgodna z konfiguracją na R1
crypto ipsec transform-set MYSET esp-3des esp-md5-hmac

! Definicja listy dostępu dla ruchu VPN
! Określenie, który ruch sieciowy ma być szyfrowany (z LAN Site B do LAN Site A)
ip access-list extended VPN-TRAFFIC
 permit ip 192.168.3.0 0.0.0.255 192.168.1.0 0.0.0.255

! Tworzenie i aplikacja mapy kryptograficznej
! Powiązanie wszystkich elementów konfiguracji VPN w jedną całość
crypto map MYMAP 10 ipsec-isakmp
 set peer 10.1.1.1            !Adres IP zdalnego routera VPN (R1)
 set transform-set MYSET      !Użycie zdefiniowanego zestawu transformacji
 match address VPN-TRAFFIC    !Powiązanie z listą dostępu definiującą ruch VPN

! Aplikacja mapy kryptograficznej do interfejsu wychodzącego (w kierunku ISP)
interface Serial0/0/1
 crypto map MYMAP
\end{lstlisting}

% \section{Weryfikacja działania VPN}
% \subsection{Polecenia weryfikacyjne}
% \begin{lstlisting}
% show crypto isakmp sa      # Wyświetla status skojarzeń bezpieczeństwa IKE (Phase 1)
% show crypto ipsec sa       # Wyświetla status tuneli IPSec (Phase 2) i statystyki ruchu
% ping 192.168.3.10 source 192.168.1.10  # Testuje połączenie end-to-end przez tunel VPN
%                                        # (przy założeniu, że hosty istnieją w obu sieciach)
% \end{lstlisting}

% \subsection{Oczekiwane wyniki}
% \begin{itemize}
% \item Status IKE: \texttt{QM\_IDLE} oznacza poprawnie nawiązane sąsiedztwo
% \item Status IPSec: \texttt{\#pkts encaps/decaps} pokazuje przepływ ruchu
% \item Udany ping między sieciami lokalnymi
% \end{itemize}

% \section{Opis kluczowych parametrów}
% \begin{tabularx}{\textwidth}{lX} % Usunięto pionowe linie
% \toprule % Zamiast \hline
% \textbf{Parametr} & \textbf{Znaczenie} \\
% \midrule % Zamiast \hline
% ISAKMP Policy 10 & Definicja parametrów IKE Phase 1 \\
% esp-3des & Szyfrowanie 3DES (168-bit) \\
% esp-md5-hmac & Algorytm integralności MD5 \\
% pre-share & Uwierzytelnianie za pomocą wspólnego klucza \\
% group 2 & Grupa Diffie-Hellman 1024-bit \\
% crypto map & Mapowanie polityk bezpieczeństwa na interfejs \\
% match address & Powiązanie ACL z mapą kryptograficzną \\
% \bottomrule % Zamiast \hline
% \end{tabularx}
\clearpage
\section{Zrzuty ekranu}

\begin{figure}[h!]
    \centering
    \includegraphics[width=0.75\textwidth]{routery/R1/ping1.png}
    \caption{Ping z R1 cz.1}
    \label{fig:adresacja_ip}
\end{figure}

\begin{figure}[ht!]
    \centering
    \includegraphics[width=0.75\textwidth]{routery/R1/ping2.png}
    \caption{Ping z R1 cz.2}
    \label{fig:adresacja_ip}
\end{figure}

\begin{figure}[ht!]
    \centering
    \begin{minipage}[t]{0.48\textwidth}
        \centering
        \includegraphics[width=\textwidth]{routery/R2/R2.png}
        \caption{Pingi do R2 oraz interfejsy}
        \label{fig:r2_ping}
    \end{minipage}
    \hfill
    \begin{minipage}[t]{0.48\textwidth}
        \centering
        \includegraphics[width=\textwidth]{routery/R3/Zrzut ekranu 2025-06-02 194011.png}
        \caption{Pingi z R3 oraz interfejsy}
        \label{fig:r3_ping}
    \end{minipage}
\end{figure}
\clearpage
\section{Podsumowanie}
Konfiguracja Site-to-Site VPN obejmuje:
\begin{enumerate}
\item Konfigurację podstawową routerów (adresacja, routing)
\item Definicję polityki ISAKMP (IKE Phase 1)
\item Konfigurację transformacji IPSec (IKE Phase 2)
\item Tworzenie list dostępu identyfikujących ruch VPN
\item Budowę i aplikację map kryptograficznych
\end{enumerate}
Poprawnie skonfigurowany tunel VPN zapewnia bezpieczną komunikację między sieciami lokalnymi poprzez nieszyfrowaną sieć publiczną.
\end{document}
