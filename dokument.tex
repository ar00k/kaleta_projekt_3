%╔════════════════════════════╗
%║	  Szablon dostosował	  ║
%║	mgr inż. Dawid Kotlarski  ║
%║		  06.10.2024		  ║
%╚════════════════════════════╝


\documentclass[12pt,twoside,a4paper,openany]{article}

    \input{preambula_pakiety.tex}
    \input{preambula_ustawienia.tex}

    %polecenia zdefiniowane w pakiecie strona_tytulowa.sty
    \title{Konfiguracja tunelu GRE typu punkt-punkt w sieci VPN}		%...Wpisać nazwę projektu...
    \author{Imie Nazwisko}
    \author{Arkadiusz Ryczek, Maciej Wójs,  }
    \authorI{Dominik Żuchowicz, Hubert Zięba}
    \authorII{Szymon Wolski, Filip Wąchała, Jakub \textit{Józef} Tokarczyk}		%jeśli są dwie osoby w projekcie to zostawiamy:    \authorII{}
		
	\uczelnia{AKADEMIA NAUK STOSOWANYCH \\W NOWYM SĄCZU}
    \instytut{Wydział Nauk Inżynieryjnych}
    \kierunek{Katedra Informatyki}
    \praca{DOKUMENTACJA PROJEKTOWA}
    \przedmiot{BEZPIECZEŃSTWO SYSTEMÓW INFORMATYCZNYCH}
    \prowadzacy{mgr inż. Jacek Kaleta}
    \rok{2025}

%definicja składni mikrotik
\usepackage{fancyvrb}
\DefineVerbatimEnvironment{MT}{Verbatim}%
{commandchars=\+\[\],fontsize=\small,formatcom=\color{red},frame=lines,baselinestretch=1,} 
\let\mt\verb
%zakonczenie definicji składni mikrotik

\usepackage{fancyhdr}    %biblioteka do nagłówka i stopki
\begin{document}

\renewcommand{\figurename}{Rys.}    %musi byc pod \begin{document}, bo w~tym miejscu pakiet 'babel' narzuca swoje ustawienia
\renewcommand{\tablename}{Tab.}     %j.w.
\thispagestyle{empty}               %na tej stronie: brak numeru
\stronatytulowa                     %strona tytułowa tworzona przez pakiet strona_tytulowa.tex

\input{tabela.tex} % tabela z danymi projektu

\pagestyle{fancy}
\newpage

%formatowanie spisu treści i~nagłówków
\renewcommand{\cftbeforesecskip}{8pt}
\renewcommand{\cftsecafterpnum}{\vskip 8pt}
\renewcommand{\cftparskip}{3pt}
\renewcommand{\cfttoctitlefont}{\Large\bfseries\sffamily}
\renewcommand{\cftsecfont}{\bfseries\sffamily}
\renewcommand{\cftsubsecfont}{\sffamily}
\renewcommand{\cftsubsubsecfont}{\sffamily}
\renewcommand{\cftparafont}{\sffamily}
%koniec formatowania spisu treści i nagłówków

\tableofcontents    %spis treści
\thispagestyle{fancy}
\newpage

%%%%%%%%%%%%%%%%%%% treść główna dokumentu %%%%%%%%%%%%%%%%%%%%%%%%%
\section{Założenia projektowe – wymagania}


\subsection{Cel projektu}
\begin{itemize}
    \item Konfiguracja tunelu GRE (Generic Routing Encapsulation) typu punkt-punkt między routerami WEST i EAST.
    \item Umożliwienie komunikacji między sieciami LAN poprzez sieć publiczną (ISP).
    \item Wykorzystanie protokołu OSPF do dynamicznego routingu wewnątrz tunelu.
\end{itemize}
\subsection{Informacje o projekcie}



\fg[\linewidth]{rys/tabela.png}{tabela z adresami}{rys-tabela}
\clearpage
\fg[\linewidth]{rys/image.png}{schemat logiczny projektu}{rys-tabela}

\clearpage
\subsection{Wymagania sprzętowe i programowe}
\begin{itemize}
    \item \textbf{Routery}: 3 × Cisco 1941 (IOS 15.2(4)M3) – WEST, EAST, ISP.
    \item \textbf{Przełączniki}: 2 × Cisco 2960 (IOS 15.0(2)).
    \item \textbf{Komputery}: 2 × PC z systemem Windows (Putty w celu połaczenia sie z terminalem routerów).
    \item \textbf{Kable}: Konsolowe, Ethernetowe, szeregowe (DCE/DTE).
\end{itemize}

\subsection{Wymagania sieciowe}
\begin{itemize}
    \item Adresacja IP zgodna z tabelą (np. 172.16.1.0/24 dla WEST, 172.16.2.0/24 dla EAST).    \item Domyślne trasy statyczne na routerach WEST i EAST wskazujące na ISP.
    \item Interfejsy tunelu (Tunnel0) z adresami:\\ 172.16.12.1 (WEST) i 172.16.12.2 (EAST).
\end{itemize}

\subsection{Bezpieczeństwo}
\begin{itemize}
    \item Podstawowe zabezpieczenia routerów (hasła, MOTD, logging synchronous).
    \item \textbf{Uwaga}: Tunel GRE jest nieszytrowany – w rzeczywistych zastosowaniach należy dodać IPSec.
\end{itemize}

\section{Wykonanie}

\subsection{Konfiguracja podstawowa}
\begin{itemize}
    \item Przypisanie adresów IP interfejsom (G0/1, S0/0/0, S0/0/1).
    \item Ustawienie domyślnych tras statycznych na WEST i EAST.
    \item Weryfikacja łączności (ping między routerami a ISP).
\end{itemize}

\subsection{Konfiguracja tunelu GRE}
\begin{itemize}    \item Definicja interfejsu \texttt{Tunnel0} na obu routerach:
    \begin{verbatim}
interface Tunnel0
ip address 172.16.12.1 255.255.255.252   # Dla WEST
tunnel source S0/0/0                      # Lub adres IP
tunnel destination 10.2.2.1               # Adres EAST
    \end{verbatim}
    \item Weryfikacja:
    \begin{itemize}
        \item \texttt{show ip interface brief} – sprawdzenie statusu tunelu.
        \item \texttt{ping 172.16.12.2} (z WEST do EAST) – test łączności.
    \end{itemize}
\end{itemize}

\subsection{Konfiguracja OSPF}
\begin{itemize}
    \item Dodanie sieci LAN i tunelu do obszaru 0:
    \begin{verbatim}
router ospf 1
network 172.16.1.0 0.0.0.255 area 0      # Dla WEST
network 172.16.12.0 0.0.0.3 area 0       # Sieć tunelu
    \end{verbatim}
    \item Weryfikacja tras: \texttt{show ip route}\\ (powinna pojawić się trasa OSPF do zdalnej sieci LAN).
\end{itemize}

\subsection{Testy end-to-end}
\begin{itemize}
    \item Ping z PC-A (172.16.1.3) do PC-C (172.16.2.3).
    \item Traceroute – ścieżka powinna przechodzić przez tunel\\ (np. \texttt{172.16.1.1 → 172.16.12.2 → 172.16.2.3}).
\end{itemize}

\section{Testy i wnioski}

\subsection{Testy}

\fg{Kaleta part 2/WEST/PING Z PCA DO PCC.png}{PING Z PCA DO PCC}{WEST-PING-Z-PCA-DO-PCC}
\begin{itemize}    \item \textbf{Pozytywne}:
    \begin{itemize}
        \item Ping przez tunel między routerami (172.16.12.1 $\leftrightarrow$ 172.16.12.2).
        \item Komunikacja PC-A $\leftrightarrow$ PC-C (potwierdza działanie OSPF w tunelu).
    \end{itemize}
    \item \textbf{Problemy i rozwiązania}:
    \begin{itemize}        \item Brak łączności: Sprawdź \texttt{tunnel source/destination} oraz czy interfejsy fizyczne są aktywne.
        \item Błędy OSPF: Upewnij się, że sieci są dodane do właściwego obszaru.
    \end{itemize}
\end{itemize}

\subsection{Wnioski}
\begin{itemize}
    \item Tunel GRE pozwala na enkapsulację ruchu między odległymi sieciami, ale \textbf{nie szyfruje danych} (wymagany IPSec).
    \item OSPF dynamicznie aktualizuje trasy, co ułatwia skalowalność\\ (np.\ dodanie nowej sieci LAN wymaga tylko aktualizacji OSPF).
\end{itemize}

\section{Konfiguracja ISP}

\begin{figure}[h]
    \centering
    \begin{minipage}{0.49\textwidth}
        \centering
        \includegraphics[width=\textwidth]{Kaleta part 2/ISP/1.png}
        \caption{Konfiguracja ISP}
        \label{fig:ISP-1}
    \end{minipage}
    \hfill
    \begin{minipage}{0.49\textwidth}
        \centering
        \includegraphics[width=\textwidth]{Kaleta part 2/ISP/2.png}
        \caption{Konfiguracja ISP}
        \label{fig:ISP-2}
    \end{minipage}
\end{figure}
\begin{figure}[h]
    \centering
    \begin{minipage}{0.49\textwidth}
        \centering
        \includegraphics[width=\textwidth]{Kaleta part 2/ISP/3.png}
        \caption{Konfiguracja ISP}
        \label{fig:ISP-3}
    \end{minipage}
    \hfill
    \begin{minipage}{0.49\textwidth}
        \centering
        \includegraphics[width=\textwidth]{Kaleta part 2/ISP/4.png}
        \caption{Konfiguracja ISP}
        \label{fig:ISP-4}
    \end{minipage}
\end{figure}

\begin{figure}[h]
    \centering
    \includegraphics[width=0.8\textwidth]{Kaleta part 2/ISP/5.png}
    \caption{Konfiguracja ISP}
    \label{fig:ISP-5}
\end{figure}
% \fg{Kaleta part 2/ISP/2.png}{Konfiguracja ISP}{ISP-2}

% \fg{Kaleta part 2/ISP/3.png}{Konfiguracja ISP}{ISP-3}
% \fg{Kaleta part 2/ISP/4.png}{Konfiguracja ISP}{ISP-4}

% \fg{Kaleta part 2/ISP/5.png}{Konfiguracja ISP}{ISP-5}
\clearpage
\section{Dokumentacja EAST I WEST}
\subsection*{Konfiguracja routera WEST}
\begin{lstlisting}[caption={Konfiguracja routera WEST}, label={lst:west-config}]
hostname WEST
!
version 15.2(4)M5
!
service password-encryption
no ip domain-lookup
!
enable secret cisco
enable password class
!
banner motd #UFO#
!
interface GigabitEthernet0/1
ip address 172.16.1.1 255.255.255.0
no shutdown
!
interface Serial0/0/0
ip address 10.1.1.1 255.255.255.252
clock rate 128000
no shutdown
!
interface Tunnel0
ip address 172.16.12.1 255.255.255.252
tunnel source Serial0/0/0
tunnel destination 10.2.2.1
no shutdown
!
router ospf 1
network 172.16.1.0 0.0.0.255 area 0
network 172.16.12.0 0.0.0.3 area 0
!
ip route 0.0.0.0 0.0.0.0 10.1.1.2
!
line con 0
logging synchronous
line vty 0 4
password cisco

\end{lstlisting}

\subsection*{Konfiguracja routera EAST}
\begin{lstlisting}[caption={Konfiguracja routera EAST}, label={lst:east-config}]
hostname EAST
!
version 15.2
!
service password-encryption
no ip domain-lookup
!
enable secret 4 tnhtc92DXBhelxjYk8LWJrPV36S2i4ntXrpb4RFmfqY
enable password 7 02050D480809
!
banner motd ^CEAST^C
!
interface GigabitEthernet0/1
ip address 172.16.2.1 255.255.255.0
no shutdown
!
interface Serial0/0/1
ip address 10.2.2.1 255.255.255.252
no shutdown
!
interface Tunnel0
ip address 172.16.12.2 255.255.255.252
tunnel source 10.2.2.1
tunnel destination 10.1.1.1
no shutdown
!
router ospf 1
network 172.16.2.0 0.0.0.255 area 0
network 172.16.12.0 0.0.0.3 area 0
!
ip route 0.0.0.0 0.0.0.0 10.2.2.2
!
line con 0
logging synchronous
line vty 0 4
password 7 05080F1C2243
login
transport input all
!
end
\end{lstlisting}

\section{Podsumowanie}
\begin{itemize}
    \item Projekt umożliwił skonfigurowanie tunelu GRE między dwoma routerami, co pozwoliło na komunikację między sieciami LAN poprzez sieć publiczną.
    \item Wykorzystanie protokołu OSPF umożliwiło dynamiczne zarządzanie trasami w tunelu.
    \item Testy potwierdziły poprawność konfiguracji i działanie tunelu GRE.
    \item Zidentyfikowano, że tunel GRE nie zapewnia szyfrowania danych, co jest istotne w kontekście bezpieczeństwa. W przyszłości zaleca się rozważenie dodania szyfrowania IPSec do tunelu GRE.
\end{itemize}

%%%%%%%%%%%%%%%%%%% koniec treść główna dokumentu %%%%%%%%%%%%%%%%%%%%%
\newpage
% \addcontentsline{toc}{section}{Literatura}
% Modified by: Maciej Wójs  
\printbibliography[heading=bibnumbered, label=Literatura, title=Literatura]

\newpage
\hypersetup{linkcolor=black}
\renewcommand{\cftparskip}{3pt}
\clearpage
\renewcommand{\cftloftitlefont}{\Large\bfseries\sffamily}
% \listoffigures
% \addcontentsline{toc}{section}{Spis rysunków}
% \thispagestyle{fancy}

% \newpage
% \renewcommand{\cftlottitlefont}{\Large\bfseries\sffamily}
% \def\listtablename{Spis tabel}
% \addcontentsline{toc}{section}{Spis tabel}\listoftables
% \thispagestyle{fancy}

% \newpage
% \renewcommand{\cftlottitlefont}{\Large\bfseries\sffamily}
% \renewcommand{\lstlistlistingname}{Spis listingów}
% \addcontentsline{toc}{section}{Spis listingów}\lstlistoflistings
% \thispagestyle{fancy}

%lista rzeczy do zrobienia: wypisuje na koñcu dokumentu, patrz: pakiet todo.sty
\todos
%koniec listy rzeczy do zrobienia
\end{document}
